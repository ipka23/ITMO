\documentclass[twocolumn]{article} % Начинаем с двух колонок и увеличиваем размер шрифта

\usepackage{graphicx} % Для вставки изображений
\usepackage{tikz} % Для рисования
\usepackage[russian]{babel} % Поддержка русского языка
\usepackage{fancyhdr}
\usepackage[left=0.5cm, right=3cm, top=3cm, paperwidth=210mm, paperheight=270mm]{geometry}


\usepackage{caption}


\pagestyle{fancy}
\fancyhf{} % Очищаем все поля

% Заголовок и номер страницы
\fancyhead[R]{КНИГА I ПРЕДЛ. II. ЗАДАЧА} % Заголовок слева
\fancyhead[C]{26} % Номер страницы справа

% Убираем линию под колонтитулом
\renewcommand{\headrulewidth}{0pt}

\begin{document}

\begin{tikzpicture}
    % оранжевый круг
    \draw[line width=0.7mm, orange] (0,0) circle (3cm);
    
    % синий круг
    \draw[line width=0.7mm, blue] (0.6,-1) circle (1.8cm);
    \node[below] at (0.6,-1) {B};
    
    % линии
    \draw[blue, line width=0.7mm] (2.95,-0.6) -- (1.2,-0.3);
    \node[above] at (1.2,-0.3) {A};
    \node[right] at (2.95,-0.6) {F};
    
    \draw[orange, line width=0.7mm] (0.3,-0.2) -- (1.2,-0.3);
    \node[left] at (0.3,-0.2) {D};
    
    \draw[yellow, line width=0.7mm] (1.7,-2.45) -- (0.6,-1);
    \node[below] at (1.7,-2.45) {E};
    \draw[dashed, navy, line width=0.7mm] (0.6,-1) -- (1.2,-0.3);
    
    \draw[navy, line width=0.7mm] (-1.2,-1) -- (0.6,-1);
    \node[left] at (-1.2,-1) {C};
    \draw[dashed, orange, line width=0.7mm] (0.6,-1) -- (0.3,-0.2);
\end{tikzpicture}

\newpage % Начинаем новый раздел с двух колонок

\begin{figure}[h]
    \centering
    \begin{tikzpicture}
        \node[left] (img) at (0, 0) {
            \includegraphics[scale=0.08]{O.png} % Укажите путь к вашему изображению
        };
        \node[right, text width=8cm, align=left] at (0, 1) { % Установите ширину текста и выравнивание
            т\textit{ данной точки}
            \begin{tikzpicture}
                \draw[thick, orange, line width=0.7mm] (0, 0) -- (0.5, 0);
                \node[above] at (0.5, 0) {A};
                \draw[thick, blue, line width=0.7mm] (0.5, 0) -- (1, 0);
            \end{tikzpicture}
            \textit{отложить прямую,
            равную данной прямой}
            \begin{tikzpicture}
                \draw[thick, navy, line width=0.7mm] (0, 0) -- (1, 0);
                \node[above] at (0, 0) {B};
                \node[above] at (1, 0) {C};
            \end{tikzpicture}.
        };
    \end{tikzpicture}
\end{figure}


\begin{center}
    Проведем 
\begin{tikzpicture}
     \draw[dashed, navy, line width=0.7mm] (0, 0) -- (1, 0);
     \node[above] at (0, 0) {A};
     \node[above] at (1, 0) {B};
\end{tikzpicture}
(пост. I),

построим
\begin{tikzpicture}
    \draw[thick, orange, line width=0.7mm] (0.3,-0.2) -- (1.2,-0.3);
    \draw[dashed, navy, line width=0.7mm] (0.6,-1) -- (1.2,-0.3);
    \draw[dashed, orange, line width=0.7mm] (0.6,-1) -- (0.3,-0.2);
\end{tikzpicture}
(пр. I.1),

продлим 
\begin{tikzpicture}
     \draw[dashed, orange, line width=0.7mm] (0, 0) -- (1, 0);
     \node[above] at (0, 0) {B};
     \node[above] at (1, 0) {D};
\end{tikzpicture}
(пост. II),

опишем
\begin{tikzpicture}
    \draw[line width=0.7mm, blue] (0,0) circle (0.5cm);
    \draw[line width=0.7mm, navy] (0, 0) -- (-0.5, 0)
    \node[left] at (-0.5, 0) {C}
    \node[right] at (0, 0) {B}
\end{tikzpicture}
(пост. III), и
\begin{tikzpicture}
    \draw[line width=0.7mm, orange] (0,0) circle (0.8cm);
    \draw[dashed, line width=0.7mm, orange] (0, 0) -- (0.10, -0.3)
    \draw[line width=0.7mm, yellow] (0.10, -0.3) -- (0.25, -0.73)
    \node[above] at (0, 0) {D}
    \node[below] at (0.25, -0.73) {E}
\end{tikzpicture}
(пост. III);

продлим
\begin{tikzpicture}
     \draw[orange, line width=0.7mm] (0, 0) -- (1, 0);
     \node[above] at (0, 0) {D};
     \node[above] at (1, 0) {A};
\end{tikzpicture}
(пост. III);

тогда искомая прямая это
\begin{tikzpicture}
     \draw[blue, line width=0.7mm] (0, 0) -- (1, 0);
     \node[above] at (0, 0) {A};
     \node[above] at (1, 0) {F};
\end{tikzpicture}
.

Поскольку
\begin{tikzpicture}
     \draw[line width=0.7mm, orange] (0, 0) -- (0.3, 0);
     \draw[dashed, line width=0.7mm, yellow] (0.3, 0) -- (1, 0);
     \node[above] at (0, 0) {E};
     \node[above] at (1, 0) {D};
\end{tikzpicture}
\textbf{=}
\begin{tikzpicture}
     \draw[line width=0.7mm, orange] (0, 0) -- (0.7, 0);
     \draw[line width=0.7mm, blue] (0.7, 0) -- (1, 0);
     \node[above] at (0, 0) {D};
     \node[above] at (1, 0) {F};
\end{tikzpicture}
(опр. 15),

и
\begin{tikzpicture}
     \draw[dashed, orange, line width=0.7mm] (0, 0) -- (1, 0);
     \node[above] at (0, 0) {B};
     \node[above] at (1, 0) {D};
\end{tikzpicture}
=
\begin{tikzpicture}
     \draw[dashed, orange, line width=0.7mm] (0, 0) -- (1, 0);
     \node[above] at (0, 0) {D};
     \node[above] at (1, 0) {A};
\end{tikzpicture}
(постр.),

\begin{tikzpicture}
    \fill (0, 0) circle (0.7pt);  
    \fill (0.12, 0) circle (0.7pt);
    \fill (0.06, 0.09) circle (0.7pt);
\end{tikzpicture}
\begin{tikzpicture}
     \draw[yellow, line width=0.7mm] (0, 0) -- (1, 0);
     \node[above] at (0, 0) {B};
     \node[above] at (1, 0) {E};
\end{tikzpicture}
=
\begin{tikzpicture}
     \draw[blue, line width=0.7mm] (0, 0) -- (1, 0);
     \node[above] at (0, 0) {A};
     \node[above] at (1, 0) {F};
\end{tikzpicture}
(акс. III),

но(опр. 15)
\begin{tikzpicture}
     \draw[navy, line width=0.7mm] (0, 0) -- (1, 0);
     \node[above] at (0, 0) {B};
     \node[above] at (1, 0) {C};
\end{tikzpicture}
=
\begin{tikzpicture}
     \draw[yellow, line width=0.7mm] (0, 0) -- (1, 0);
     \node[above] at (0, 0) {B};
     \node[above] at (1, 0) {E};
\end{tikzpicture}
=
\begin{tikzpicture}
     \draw[blue, line width=0.7mm] (0, 0) -- (1, 0);
     \node[above] at (0, 0) {A};
     \node[above] at (1, 0) {F};
\end{tikzpicture}
.

\begin{tikzpicture}
    \fill (0, 0) circle (0.7pt);  
    \fill (0.12, 0) circle (0.7pt);
    \fill (0.06, 0.09) circle (0.7pt);
\end{tikzpicture}
\begin{tikzpicture}
     \draw[blue, line width=0.7mm] (0, 0) -- (1, 0);
     \node[above] at (0, 0) {A};
     \node[above] at (1, 0) {F};
\end{tikzpicture}
проведенная из данной точки 

(\begin{tikzpicture}
    \draw[thick, orange, line width=0.7mm] (0, 0) -- (0.5, 0);
    \node[above] at (0.5, 0) {A};
    \draw[thick, blue, line width=0.7mm] (0.5, 0) -- (1, 0);
\end{tikzpicture})
равна данной прямой
\begin{tikzpicture}
     \draw[тфмн, line width=0.7mm] (0, 0) -- (1, 0);
     \node[above] at (0, 0) {B};
     \node[above] at (1, 0) {C};
\end{tikzpicture}
(акс. I).

\begin{flushright}
    ч. т. д.
\end{flushright}

\end{center}
\end{document}
